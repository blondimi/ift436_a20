\documentclass{article}

%% Mise en pages
\usepackage{fontspec}                 %% >> Pour compiler
\setmainfont{XCharter}                %%                 avec xelatex <<
%\usepackage[utf8]{inputenc}          %% >> Pour compiler avec pdflatex <<
\usepackage[french]{babel}            %% Français
\NoAutoSpacing                        %% Retire l'espacement devant «:»
\usepackage{geometry}                 %% Agrandir les marges
\geometry{left=2.5cm, right=2.5cm, top=2.25cm}
\setlength{\parskip}{\baselineskip}   %% Ajouter espacement entre lignes
\setlength{\parindent}{0pt}           %% Retirer indentation
\usepackage{fancyhdr}                 %% En-têtes/pieds de page
\pagestyle{fancy}                     %%
\fancyhf{}                            %%
\fancyhead[R]{\session}               %%
\fancyhead[L]{\sigle~--~\coursentete} %%
\fancyhead[C]{\evaluation}            %%
\fancyfoot[C]{\thepage}               %%

%% Packages
\usepackage[dvipsnames]{xcolor}       %% Plus de couleurs
\usepackage{amsmath,amssymb}          %% Maths
\usepackage{mathtools}                %% \coloneqq
\usepackage{enumerate}                %% Extension de \begin{enumerate}
\usepackage{exsheets}                 %% Environnements d'exercices
\newcommand*\pointsformat[1]{{\scriptsize#1}}
\SetupExSheets{                       %%
  question/name        = {Question},  %%
  points/name          = {\scriptsize pt/s},
  points/bonus-name    = {\scriptsize pt/s},
}                                     %%
\usepackage{tikz}                     %% Dessins
\usetikzlibrary{shapes, positioning}  %%
\usepackage{hyperref}                 %% Hyperliens
\hypersetup{                          %%
    colorlinks,                       %%
    linkcolor={blue!70!black},        %%
    citecolor={blue!70!black},        %%
    urlcolor={red!70!black}           %%
}                                     %%
\usepackage[noline, onelanguage, french,
  noend]{algorithm2e}                 %% Algorithmes
\usepackage{fontawesome}              %% Icônes
\usepackage{multicol}

%% Virgules francophones dans pointage
%%  (tiré de https://tex.stackexchange.com/a/396982/68991)
\usepackage[locale=FR, detect-weight]{siunitx}
\ExplSyntaxOn
\cs_set_protected:Npn \exsheets_num:n #1
 {
  \num{\fp_eval:n{#1}}
 }
\ExplSyntaxOff

%% Bonus
\newcommand{\avance}{{\large$\bigstar$}}         %% Symbole pour bonus
\newcommand{\avancepts}{{\scriptsize$\bigstar$}} %%

%% Algorithmes
\newcommand{\N}{\mathbb{N}}                      %% Entiers naturels
\renewcommand{\O}{\mathcal{O}}                   %% Notation O(·)
\SetKwRepeat{Do}{faire}{tant que}                %% Structure do..while
\newcommand{\algskip}{\vspace{3pt}}              %% Espacement vertical
\definecolor{colComment}{rgb}{0.13, 0.55, 0.13}  %% Commentaires de couleur
\newcommand\commentstyle[1]{\ttfamily\textcolor{colComment}{#1}}
\SetCommentSty{commentstyle}                     %%

%% Couleurs
\colorlet{colEmph}{cyan!80!blue}
\colorlet{colEmph2}{magenta!80!black}
\colorlet{colEmph3}{green!70!black}
\colorlet{colEmph4}{blue!80!purple}
\colorlet{colEmph5}{orange}

\colorlet{colBad}{red}
\colorlet{colWarning}{orange}

%% Commande de titre
\newcommand{\maketitlecustom}{%
  \begin{center}
    \begin{tabular}{c}
      \Large \sigle~--~\cours         \\[3pt]
      \large Université de Sherbrooke \\[10pt]
      \LARGE \textbf{\evaluation}     \\[10pt]
    \end{tabular}
    \begin{tabular}{lp{7.35cm}}
      Enseignant:     & \enseignant   \\
      Date de remise: & \remise       \\
      À réaliser:     & \equipes      \\
      Modalités:      & \modalites    \\
      Bonus:          & \bonust       \\
      Pointage:       & max.\ \pointssum*~points + \bonussum*~points bonus \\
      Consignes       & \consignes
    \end{tabular}
  \end{center}
}

%% Information de l'évaluation
\newcommand{\evaluation}{Devoir 3}
\newcommand{\remise}{mercredi 11 novembre 2020 à 10h29}
\newcommand{\modalites}{remettre en ligne sur \href{http://opus.dinf.usherbrooke.ca/}{Turnin} dans un fichier PDF}
\newcommand{\equipes}{en équipe de deux {\scriptsize ou individuellement}}
\newcommand{\bonust}{les questions bonus sont indiquées par \avance{}}
\newcommand{\sigle}{IFT436}
\newcommand{\cours}{Algorithmes et structures de données}
\newcommand{\coursentete}{Algorithmes et struct.\ de données}
\newcommand{\session}{Automne 2020}
\newcommand{\enseignant}{Michael Blondin}
\newcommand{\consignes}{commentez et décrivez les grandes lignes de
  vos algorithmes afin de faciliter leur compréhension}

%% Contenu
\begin{document}

\maketitlecustom

\SetupExSheets{
  question/name        = {Question},
  points/name          = {\scriptsize pt/s},
  points/bonus-name    = {\scriptsize pt/s},
  points/number-format = \pointsformat,
  points/bonus-format  = \pointsformat,
}

\vspace*{-10pt}

\begin{question}
  \begin{enumerate}[(a)]
    \setlength\itemsep{15pt}
    
  \item Soit $\mathcal{P}_n$ l'ensemble des pavages, d'une grille $4
    \times n$, constitués de tuiles de cette forme (sans
    rotations):\bigskip
    \begin{center}
      \hfill
      \hfill
      \hfill
      \begin{tikzpicture}[scale=0.35, transform shape]
        \fill[colEmph2] (0, 0) -- (0, 4) -- (1, 4) -- (1, 0) -- cycle;%
        \draw[white] (0, 1) -- (1, 1);%
        \draw[white] (0, 2) -- (1, 2);%
        \draw[white] (0, 3) -- (1, 3);%
      \end{tikzpicture}%
      \hfill
      \begin{tikzpicture}[scale=0.35, transform shape]
        \fill[colEmph4] (0, 0) -- (0, 2) -- (1, 2) -- (1, 0) -- cycle;%
        \draw[white] (0, 1) -- (1, 1);%
      \end{tikzpicture}%
      \hfill
      \begin{tikzpicture}[scale=0.35, transform shape]
        
        \fill[colEmph5] (0, 0) -- (0, 1) -- (1, 1) -- (1, 3) -- (0, 3)
        -- (0, 4) -- (2, 4) -- (2, 0) -- cycle;
        
        \draw[white] (0, 1) -- (2, 1);%
        \draw[white] (0, 2) -- (2, 2);%
        \draw[white] (0, 3) -- (2, 3);%
        \draw[white] (1, 0) -- (1, 4);%
      \end{tikzpicture}%
      \hfill
      \begin{tikzpicture}[scale=0.35, transform shape]
        
        \fill[colEmph] (0, 0) -- (0, 2) -- (1, 2) -- (1, 4) -- (2, 4) --
        (2, 0) -- cycle;
        
        \draw[white] (0, 1) -- (2, 1);%
        \draw[white] (0, 2) -- (2, 2);%
        \draw[white] (0, 3) -- (2, 3);%
        \draw[white] (1, 0) -- (1, 4);%
      \end{tikzpicture}%
      \hfill
      \begin{tikzpicture}[scale=0.35, transform shape]
        
        \fill[colEmph3] (1, 0) -- (1, 2) -- (0, 2) -- (0, 4) -- (2, 4)
        -- (2, 0) -- cycle;
        
        \draw[white] (0, 1) -- (2, 1);%
        \draw[white] (0, 2) -- (2, 2);%
        \draw[white] (0, 3) -- (2, 3);%
        \draw[white] (1, 0) -- (1, 4);%
      \end{tikzpicture}%
      \hfill
      \hfill
      \hfill
      \hfill
      \hfill
  \end{center}\bigskip\medskip

  \begin{enumerate}[(i)]
    \setlength\itemsep{10pt}
    
  \item Donnez \marginpar{\addpoints{4}} une récurrence linéaire $t$
    telle que $t(n) = |\mathcal{P}_n|$ pour tout $n \in \N_{\geq
      1}$. Expliquez comment vous avez obtenu votre récurrence.

  \item Appliquez \marginpar{\addpoints{5}} la méthode de résolution
    de récurrences linéaires présentée en classe afin d'obtenir une
    forme close pour $t$. Laissez une trace de votre démarche.

  \item Donnez \marginpar{\addpoints{3}} la complexité asymptotique de
    $t$ aussi précisément que possible avec la notation
    $\Theta$. Justifiez.
  \end{enumerate}

  
  \item Au tout premier cours de la session, nous avons brièvement vu
    cet algorithme:\smallskip

    \begin{algorithm}[H]
      \DontPrintSemicolon
      \KwIn{séquence $s$ de $n \in \N_{\geq 1}$ entiers}
      \Sortie{$\max\{s[i] : 1 \leq i \leq n\}$}
      
      \algskip
      
      \SetKwFunction{fmax}{max-rec}
      \SetKwProg{myproc}{}{}{}
      
      \myproc{\fmax{$s$}:}{
        \If{$n = 1$}{\Return{$s[1]$}}
        \Else{
          $m\phantom{'} \leftarrow \fmax{$s[1 : n \div 2]$}$\;
          $m' \leftarrow \fmax{$s[n \div 2 + 1 : n]$}$\;
          
          \algskip
          \lIf{$m > m'$}{\Return{$m$}}
          \lElse{\hspace*{44pt}\Return{$m'$}}
        }
      }
    \end{algorithm}
    
    \begin{enumerate}[(i)]
      \setlength\itemsep{10pt}
      
    \item Donnez \marginpar{\addpoints{4}} une récurrence qui décrit le
      temps d'exécution \emph{exact} de $\fmax$ par rapport à $n = |s|$
      (donc avec les constantes). Considérez ces opérations comme
      élémentaires: \setlength\columnsep{-4cm}
      \begin{multicols}{2}
        \begin{itemize}
          \setlength\itemsep{1pt}
          
        \item comparaison;
          
        \item addition;
          
        \item division;
          
        \item affectation;
          
        \item accès au $i^\text{ème}$ élément d'une séquence;
          
        \item obtention d'une sous-séquence $s[i : j]$ (\og
          \href{https://en.wikipedia.org/wiki/Array_slicing}{
            \emph{slicing}} \fg{}).

    \end{itemize}
    \end{multicols}

    \item L'algorithme $\fmax$ est-il \marginpar{\addpoints{4}} plus
      rapide (asymptotiquement) qu'un algorithme itératif usuel pour
      le calcul du maximum d'une séquence? Justifiez. Si cela vous
      aide, vous pouvez supposer que $n$ est une puissance de deux.
  \end{enumerate}
    
  \end{enumerate}
\end{question}

\pagebreak

\begin{question}
  \begin{enumerate}[(a)]
    \setlength\itemsep{7pt}

  \item Imaginons \marginpar{\addpoints{10}} un jeu où un diamant se
    trouve au sommet d'une colline de blocs. Vous devez atteindre le
    diamant avec un projectile. Vous avez accès à une decription de la
    colline sous forme d'une séquence $c$ où $c[i] \in \N$ indique le
    nombre de blocs de la $i^\text{ème}$ colonne de la colline. Par
    exemple, $c = [1, 2, 4, 6, 5, 4, 3, 1, 0]$ décrit cette
    colline:\bigskip
    \begin{center}
      \begin{tikzpicture}[very thick, scale=0.58, transform shape]
        \colorlet{colBlocks}{brown!20}
        
        \foreach \y in {0,...,0}
        \draw[fill=colBlocks] (0, \y) -- (0, \y+1) -- (1, \y+1)
        -- (1, \y) -- cycle;
      
        \foreach \y in {0,...,1}
        \draw[fill=colBlocks] (1, \y) -- (1, \y+1) -- (2, \y+1)
        -- (2, \y) -- cycle;
      
        \foreach \y in {0,...,3}
        \draw[fill=colBlocks] (2, \y) -- (2, \y+1) -- (3, \y+1)
        -- (3, \y) -- cycle;
        
        \foreach \y in {0,...,5}
        \draw[fill=colBlocks] (3, \y) -- (3, \y+1) -- (4, \y+1)
        -- (4, \y) -- cycle;
        
        \foreach \y in {0,...,4}
        \draw[fill=colBlocks] (4, \y) -- (4, \y+1) -- (5, \y+1)
        -- (5, \y) -- cycle;

        \foreach \y in {0,...,3}
        \draw[fill=colBlocks] (5, \y) -- (5, \y+1) -- (6, \y+1)
        -- (6, \y) -- cycle;

        \foreach \y in {0,...,2}
        \draw[fill=colBlocks] (6, \y) -- (6, \y+1) -- (7, \y+1)
        -- (7, \y) -- cycle;

        \foreach \y in {0,...,0}
        \draw[fill=colBlocks] (7, \y) -- (7, \y+1) -- (8, \y+1)
        -- (8, \y) -- cycle;

        \draw (8, 0) -- (9, 0);
        
        \node[scale=2.25, colEmph!90!blue] at (3.5, 6.5) {\faDiamond};

        \foreach \x in {1,...,9}
        \node[font=\large] at (\x - 0.5, -0.5) {\textsf{\x}};

        \foreach \y in {1,...,7}
        \node[font=\large] at (-0.5, \y-0.5) {\textsf{\y}};

      \end{tikzpicture}
    \end{center}\bigskip\smallskip
    
    Formellement, une séquence $c$ de $n \in \N_{\geq 3}$ nombres
    naturels forme une \emph{colline} s'il existe $i \in [n]$ tel que
    $$c[1] < c[2] < \cdots < c[i - 1] < c[i] > c[i + 1] > \cdots >
    c[n-1] > c[n].$$ Cet indice $i$ indique la colonne du sommet de
    $c$. Le diamant se trouve sur le bloc au sommet de la colline, par
    ex.\ à la position $(4, 7)$ pour la colline ci-dessus. Vous n'avez
    qu'un seul projectile et très peu de temps pour le lancer. Donnez
    donc un algorithme qui résout ce problème en temps $\O(\log n)$:
  
    \begin{quote}
      \vspace*{6pt}
      \begin{tabular}{lp{10cm}}
        \textsc{Entrée}: & séquence $c$ décrivant une colline de $n \in
        \N_{\geq 3}$ colonnes \\[2pt]
        
        \textsc{Sortie}: & position $(i, j)$ du diamant de la colline $c$
      \end{tabular}
      \vspace*{5pt}
    \end{quote}

  \item Imaginons \marginpar{\addpoints{10}} un jeu où des gardes
    munis de sarbacanes défendent une forteresse. Chaque garde se
    situe à une \emph{po\-si\-tion} $(x, y) \in \N^2$ et surveille la
    région $\{(a, b) \in \N^2 : a \leq x \land b \leq y\}$. Par
    exemple, huit gardes aux positions $[(6, 4), (4, 2), (8, 1), (2,
      6), (3, 5), (7, 3), (1, 3), (2, 1)]$ surveillent collectivement
    cette région globale:

    \begin{center}
      \input{forteresse}
    \end{center}\medskip

    Vous désirez améliorer votre forteresse, mais n'avez plus de pièce
    d'or. Vous désirez donc retirer autant de gardes que possible afin
    d'obtenir des pièces en échange, et ce \emph{sans changer la
      ré\-gion globale} surveillée par les gardes. Par exemple, on
    peut retirer trois gardes au maximum dans le scénario
    ci-dessus. Donnez donc un algorithme qui résout ce problème en
    temps $\O(n \log n)$:

    \begin{quote}
      \vspace*{6pt}
      \begin{tabular}{lp{10cm}}
        \textsc{Entrée}: & séquence $p$ des positions des $n \in
        \N_{\geq 1}$ gardes \\[2pt]
        
        \textsc{Sortie}: & plus grand nombre de gardes qu'on peut
        retirer de $p$ sans changer la région globale surveillée
        collectivement
      \end{tabular}
      \vspace*{5pt}
    \end{quote}
    \emph{Indice: divisez le royaume pour régner!}
    
  \end{enumerate}
\end{question}

\pagebreak

\begin{question}
  Donnez \marginpar{\addpoints{10}} un algorithme qui résout ce
  problème en temps $\Theta(n^{\log_3 6})$ dans le pire cas:
  \begin{quote}
    \vspace*{6pt}
    \begin{tabular}{ll}
      \textsc{Entrée}: & $x \in \N$ représenté avec $n$ chiffres en
      base $10$ \\[2pt]
      
      \textsc{Sortie}: & $x^2$
    \end{tabular}
    \vspace*{5pt}
  \end{quote}

  Expliquez pourquoi il fonctionne en temps $\O(n^{\log_3 6})$. Vous
  n'avez pas à justifier l'appartenance à $\Omega(n^{\log_3 6})$.

  \emph{Précision: nous faisons ces hypothèses (comme en
  classe):}\medskip
  \begin{itemize}
    \setlength\itemsep{5pt}
    
  \item \emph{L'addition et la soustraction de deux nombres d'au plus $k$
    chiffres s'effectue en temps $\O(k)$;}

  \item \emph{Le décalage et la troncation de $k$ chiffres d'un nombre
    s'effectue en temps $\O(k)$.}
  \end{itemize}\bigskip

  \emph{Indice: inspirez-vous de l'algorithme de multiplication rapide
    vu en classe mais en divisant autrement.}

  \vspace*{1.5cm}
  
  \avance{} Donnez \marginpar{\avancepts{} \addbonus{3}} un algorithme
  plus efficace pour le cas spécifique d'un nombre $x$ composé d'un
  seul chiffre répété, par ex.\ $x = 3333 3333 33$. Plus précisément,
  donnez un algorithme qui résout ce problème en temps $\O(n)$:
  \begin{quote}
    \vspace*{6pt}
    \begin{tabular}{ll}
      \textsc{Entrée}: & $x \in \N$ composé d'un seul chiffre en
      base $10$ répété $n$ fois \\[2pt]
      
      \textsc{Sortie}: & $x^2$
    \end{tabular}
    \vspace*{5pt}
  \end{quote}  
  Expliquez brièvement le fonctionnement de votre algorithme et
  pourquoi il s'exécute en temps $\O(n)$.
\end{question}

\end{document}
