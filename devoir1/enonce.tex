\PassOptionsToPackage{table}{xcolor}
\documentclass{article}

%% Mise en pages
\usepackage{fontspec}                 %% >> Pour compiler
\setmainfont{XCharter}                %%                 avec xelatex <<
%\usepackage[utf8]{inputenc}          %% >> Pour compiler avec pdflatex <<
\usepackage[french]{babel}            %% Français
\NoAutoSpacing                        %% Retire l'espacement devant «:»
\usepackage{geometry}                 %% Agrandir les marges
\geometry{left=2.5cm, right=2.5cm, top=2.25cm}
\setlength{\parskip}{\baselineskip}   %% Ajouter espacement entre lignes
\setlength{\parindent}{0pt}           %% Retirer indentation
\usepackage{fancyhdr}                 %% En-têtes/pieds de page
\pagestyle{fancy}                     %%
\fancyhf{}                            %%
\fancyhead[R]{\session}               %%
\fancyhead[L]{\sigle~--~\coursentete} %%
\fancyhead[C]{\evaluation}            %%
\fancyfoot[C]{\thepage}               %%

%% Packages
\usepackage{amsmath,amssymb}          %% Maths
\usepackage{mathtools}                %% \coloneqq
\usepackage{multicol}                 %% \begin{multicols}
\usepackage{enumerate}                %% Extension de \begin{enumerate}
\usepackage{exsheets}                 %% Environnements d'exercices
\newcommand*\pointsformat[1]{{\scriptsize#1}}
\SetupExSheets{                       %%
  question/name        = {Question},  %%
  points/name          = {\scriptsize pt/s},
  points/bonus-name    = {\scriptsize pt/s},
}                                     %%
\usepackage{tikz}                     %% Dessins
\usetikzlibrary{patterns}             %%
\usepackage{hyperref}                 %% Hyperliens
\hypersetup{                          %%
    colorlinks,                       %%
    linkcolor={blue!70!black},        %%
    citecolor={blue!70!black},        %%
    urlcolor={red!70!black}           %%
}                                     %%
\usepackage[noline, onelanguage, french,
            noend]{algorithm2e}       %% Algorithmes

%% Virgules francophones dans pointage
%%  (tiré de https://tex.stackexchange.com/a/396982/68991)
\usepackage[locale=FR, detect-weight]{siunitx}
\ExplSyntaxOn
\cs_set_protected:Npn \exsheets_num:n #1
 {
  \num{\fp_eval:n{#1}}
 }
\ExplSyntaxOff

%% Bonus
\newcommand{\avance}{{\large$\bigstar$}}         %% Symbole pour bonus
\newcommand{\avancepts}{{\scriptsize$\bigstar$}} %%

%% Algorithmes
\newcommand{\A}{\mathcal{A}}                     %% Nom générique d'algo.
\newcommand{\B}{\mathcal{B}}                     %% Nom générique alternatif
\renewcommand{\O}{\mathcal{O}}                   %% Notation O(·)
\newcommand{\F}{\mathcal{F}}                     %% Ensemble de fonctions
\newcommand{\algskip}{\vspace{3pt}}              %% Espacement vertical
\definecolor{colComment}{rgb}{0.13, 0.55, 0.13}  %% Commentaires de couleur
\newcommand\commentstyle[1]{\ttfamily\textcolor{colComment}{#1}}
\SetCommentSty{commentstyle}                     %%

%% Maths
\newcommand{\defeq}{\coloneqq}                   %% Définition (opérateur)
\newcommand{\N}{\mathbb{N}}                      %% Entiers naturels

%% Couleurs
\colorlet{colEmph}{cyan!80!blue}
\colorlet{colEmph2}{magenta!80!black}
\colorlet{colEmph3}{green!70!black}
\colorlet{colEmph4}{blue!80!purple}
\colorlet{colEmph5}{orange}

%% Commande de titre
\newcommand{\maketitlecustom}{%
  \begin{center}
    \begin{tabular}{c}
      \Large \sigle~--~\cours         \\[3pt]
      \large Université de Sherbrooke \\[10pt]
      \LARGE \textbf{\evaluation}     \\[10pt]
    \end{tabular}
    \begin{tabular}{lp{6.25cm}}
      Enseignant:     & \enseignant   \\
      Date de remise: & \remise       \\
      À réaliser:     & \equipes      \\
      Modalités:      & \modalites    \\
      Bonus:          & \bonust       \\
      Pointage:       & max.\ \pointssum*~points + \bonussum*~points bonus
    \end{tabular}
  \end{center}
}

%% Information de l'évaluation
\newcommand{\evaluation}{Devoir 1}
\newcommand{\remise}{mercredi 23 septembre 2020 à 10h29}
\newcommand{\modalites}{remettre en ligne sur \href{http://opus.dinf.usherbrooke.ca/}{Turnin}}
\newcommand{\equipes}{en équipe de deux {\scriptsize ou individuellement}}
\newcommand{\bonust}{les questions bonus sont indiquées par \avance{}}

\newcommand{\sigle}{IFT436}
\newcommand{\cours}{Algorithmes et structures de données}
\newcommand{\coursentete}{Algorithmes et struct.\ de données}
\newcommand{\session}{Automne 2020}
\newcommand{\enseignant}{Michael Blondin}

%% Contenu
\begin{document}

\maketitlecustom

\SetupExSheets{
  question/name        = {Question},
  points/name          = {\scriptsize pt/s},
  points/bonus-name    = {\scriptsize pt/s},
  points/number-format = \pointsformat,
  points/bonus-format  = \pointsformat,
}

\vspace*{0.5cm}

%% Question 1
\SetupExSheets{counter-format=qu[1]:~{rédaction et analyse d'algorithmes}}

\newcommand{\salle}[3]{%
  \raisebox{2.5pt}{%
    \fcolorbox{black}{#3}{%
      \makebox[#2ex][c]{$\underset{#2~mins.}{\text{salle $#1$}}$}%
    }%
  }\hspace*{-0.5pt}%
}

\begin{question}\medskip
  L'escouade étudiante de l'université nettoie les salles de classe
  chaque matin. En général, il y a $m \in \N_{>0}$ salles à nettoyer
  par $n \in \N_{>0}$ personnes. Le temps requis pour laver la
  $i^\text{ème}$ salle est de $t[i] \in \N_{>0}$ minutes. Ainsi, une
  personne qui nettoierait l'entiereté du campus mettrait $t[1] + t[2]
  + \cdots + t[m]$ minutes pour y arriver. Comme l'escouade est
  normalement constituée de plusieurs personnes, jusqu'à $n$ salles
  peuvent être traitées simultanément. Par soucis de distanciation, il
  n'est pas permis d'assigner plus d'une personne à une même~salle.

  Dans le but de compléter le nettoyage aussi rapidement que possible,
  l'université considère l'approche suivante. Les personnes sont
  numérotées de $1$ à $n$, et les salles sont distribuées à tour de
  rôle à la personne qui détient la plus petite charge de travail. Si
  plusieurs personnes possèdent une charge minimale, alors la personne
  avec le plus petit numéro d'identification reçoit la salle. Par
  exemple, considérons $n = 3$ personnes et ces $m = 5$~salles:
  \begin{center}
    \begin{tabular}{|c|>{\itshape}l|c|} \hline
      \multicolumn{2}{|c|}{\textbf{\emph{Salle}}} &
      \textbf{\emph{Temps requis}} \\

      \rowcolor{colEmph2!20}
      \hline $1$ & D2-1060 & 21 mins. \\

      \rowcolor{colEmph4!20}
      \hline $2$ & D7-2023 & 20 mins. \\
      
      \rowcolor{colEmph5!20}
      \hline $3$ & D4-2011 & 16 mins. \\

      \rowcolor{colEmph!20}
      \hline $4$ & D7-2021 & 18 mins. \\

      \rowcolor{colEmph3!20}
      \hline $5$ & D3-2029 & 13 mins. \\

      \hline
    \end{tabular}
  \end{center}
  Initialement, chaque personne possède une charge de travail de 0
  minute. La première personne se fait donc assigner la salle 1. Les
  trois personnes possèdent maintenant une charge de travail de 21, 0
  et 0 minutes, respectivement. Ainsi, la deuxième personne reçoit la
  salle 2, puis, la troisième personne reçoit la salle 3. À ce stade,
  elles possèdent une charge de 21, 20 et 16 minutes,
  respectivement. La troisième personne possède donc la plus petite
  charge de travail et se fait ainsi assigner la salle 4. Les charges
  sont donc maintenant de 21, 20 et 34 minutes. La salle 5 est ainsi
  assignée à la deuxième personne. Nous obtenons cette distribution:
  \begin{center}
    \begin{tabular}{cl}
      \textbf{\emph{Personne 1:}} &
      \salle{1}{21}{colEmph2!20} \\[15pt]

      \textbf{\emph{Personne 2:}} &
      \salle{2}{20}{colEmph4!20}\salle{5}{13}{colEmph3!20} \\[15pt]

      \textbf{\emph{Personne 3:}} &
      \salle{3}{16}{colEmph5!20}\salle{4}{18}{colEmph!20}
    \end{tabular}
  \end{center}
  Ainsi, le nettoyage sera complété en 34 minutes.

  \pagebreak

  \begin{enumerate}[(a)]
  \setlength\itemsep{15pt}

  \item Décrivez \marginpar{\addpoints{6}} la procédure de
    distribution des salles sous forme de \emph{pseudocode}. Spécifiez
    le format de l'entrée et de la sortie de l'algorithme. Seules les
    séquences (tableaux et listes) sont permises comme structures de
    données. Vous ne pouvez pas invoquer des algorithmes existants en
    \og boîtes noires \fg{}. Sur l'entrée de la page précédente,
    l'algorithme devrait retourner $\left[[1], [2, 5], [3,
        4]\right]$.\label{itm:algo}

  \item Analysez \marginpar{\addpoints{5}} le temps d'exécution de la
    procédure décrite en~\eqref{itm:algo} dans le \emph{pire cas} et
    le \emph{meilleur cas} à la manière du diaporama
    d'introduction. Vous devez compter le nombre d'opérations
    élémentaires en fonction de $m$ et $n$ (et non pas utiliser la
    notation $\O$). Indiquez quelles opérations sont considérées
    élémentaires. Laissez une trace de vos calculs.

  \item Prouvez \marginpar{\addpoints{3}} que $34$ minutes est bel et
    bien la durée minimale pour l'exemple de la page précédente;
    autrement dit, prouvez qu'il n'existe \emph{aucune} distribution
    des cinq salles qui donne une durée \emph{inférieure} à $34$
    minutes. Donnez un argument plus concis qu'une simple énumération
    de toutes les distributions possibles.

  \item Montrez \marginpar{\addpoints{3}} que la procédure décrite
    en~\eqref{itm:algo} ne retourne pas toujours une distribution de
    durée minimale.

  \item Donnez \marginpar{\addpoints{5}} un algorithme qui
    retourne \emph{toujours} une distribution de durée minimale, en
    complétant le pseudocode ci-dessous. Ne vous souciez pas de son
    efficacité.\label{itm:algo2}\bigskip
    
    \hspace*{20pt}%
    \begin{algorithm}[H]
      \DontPrintSemicolon
      \LinesNumbered
      
      \nl\tcp{à compléter}
      \algskip
      \For{toute séquence $s$ de $m$ éléments
                 parmi $\{1, 2, \ldots, n\}$\label{ln:loop}}{
        \algskip
        \textcolor{white}{ }\tcp{à compléter} %% Hack: ligne vide
      }
      \algskip
      \textcolor{white}{ }\tcp{à compléter}   %% Hack: ligne vide
    \end{algorithm}
    \bigskip
    \emph{Par exemple, pour $m = 3$ et $n = 2$, la boucle principale
      énumère ces séquences:} \[[1, 1, 1],\ [2, 1, 1],\ [1, 2,
      1],\ [2, 2, 1],\ [1, 1, 2],\ [2, 1, 2],\ [1, 2, 2],\ [2, 2,
      2].\]

  \item Combien \marginpar{\addpoints{3}} de tours de la boucle
    principale sont effectués dans le pire cas dans l'algorithme
    obtenu en~\eqref{itm:algo2}? Autrement dit, combien de fois est-ce
    que la ligne~\ref{ln:loop} est atteinte dans le pire cas?
    \\[75pt]

  \item[\avance{}] Donnez \marginpar{\avancepts{} \addbonus{2.5}} une
    implémentation, sous forme de pseudocode, de la procédure décrite
    en~\eqref{itm:algo}, qui fonctionne en temps $\O(n + m \log n)$
    dans le pire cas. Vous pouvez cette fois utiliser des structures
    de données avancées, dont celles introduites en IFT339.
  \end{enumerate}
\end{question}

\pagebreak

%% Question 2
\SetupExSheets{counter-format=qu[1]:~{conception d'algorithmes}}

\begin{question}\medskip
  Dû à la logistique de nettoyage, l'université cherche également à
  minimiser le nombre de salles utilisées dans une journée donnée. Par
  exemple, considérons une journée qui compterait trois cours à ces
  périodes respectives: 8h30 à 10h30, 9h30 à 11h30 et 14h30 à
  16h30. Bien que ces cours pourraient avoir lieu dans trois salles
  distinctes, il est possible d'utiliser deux salles le matin et d'en
  réutiliser une l'après-midi. De façon générale, nous cherchons ainsi
  à minimiser le nombre de salles en fonction d'un horaire donné. On
  ignore ici les tailles physiques et le nombre d'inscriptions.

  Formellement, un \emph{cours} est une paire $(i, j) \in \N^2$, telle
  que $i < j$, où $i$ représente le moment où le cours débute et $j$
  représente celui où il se termine. Deux cours $(i, j)$ et $(k,
  \ell)$ peuvent avoir lieu dans la même salle si $j \leq k$;
  autrement dit, s'ils ne se chevauchent pas. Un \emph{horaire} est
  une séquence finie de cours. Un horaire est réalisable dans $m$
  salles si l'on peut assigner à chaque cours une salle parmi $m$
  choix, sans créer de conflit. Par exemple, considérons l'horaire $H
  = [(0, 2), (1, 3), (1, 4), (9, 13), (7, 8), (8, 11)]$ illustré
  ci-dessous. Celui-ci peut être réalisé avec trois salles, par
  ex.\ les salles $[1, 2, 3, 2, 3, 1]$. Il est \emph{impossible} de
  réaliser $H$ avec moins de salles puisqu'il y a trois cours en
  simultané au moment $1$. Ainsi, la sortie attendue pour cet horaire
  $H$ est $3$.

  \begin{center}
    \begin{tikzpicture}[thick]
      %% Lignes verticales
      \foreach \x in {0,...,13} {
        \draw[ultra thin, black!25] (\x, 1 - 0.25) -- (\x, 4 - 0.25);
        \node[font=\sffamily, yshift=-10pt] at (\x, 1 - 0.25) {\x};
      }

      %% Blocs rectangulaires
      \newcommand{\bloc}[5]{%
        \draw[#4, fill=#4, pattern=#5, pattern color=#4]
        (#1, #3) rectangle (#2, #3 + 0.5);
      }

      \bloc{0}{2}{1}{colEmph}{north west lines}
      \bloc{1}{3}{2}{black}{crosshatch}
      \bloc{1}{4}{3}{colEmph3}{north east lines}
      \bloc{9}{13}{2}{colEmph4}{crosshatch}
      \bloc{7}{8}{3}{colEmph5}{north east lines}
      \bloc{8}{11}{1}{colEmph2}{north west lines}
    \end{tikzpicture}
  \end{center}  

  \begin{enumerate}[(a)]
  \setlength\itemsep{15pt}

  \item Quel est \marginpar{\addpoints{1}} le plus petit nombre et le
    plus grand nombre de salles pouvant être requis pour un horaire de
    $n$ cours? Dans les deux cas, donnez un exemple générique où cela
    se produit.

  \item Combien \marginpar{\addpoints{1}} y a-t-il de façons
    d'assigner $m$ salles à $n$ cours (en incluant celles qui peuvent
    créer des conflits)? Un algorithme qui génèrerait toutes ces
    assignations fonctionnerait-il en temps polynomial?

  \item Quel est \marginpar{\addpoints{1}} le nombre minimal de salles
    permettant de réaliser l'horaire ci-dessous? \[[(0, 5), (1, 12),
      (8, 11), (3, 8), (12, 14), (13, 15), (18, 20), (1, 16)]\]

  \item Donnez un \marginpar{\addpoints{7}} algorithme, sous forme de
    pseudocode, qui résout ce problème:
    \begin{quote}
      \smallskip
      \begin{tabular}{ll}
        \textsc{Entrée}: & un horaire $H = [(i_1, j_1), \ldots, (i_n,
          j_n)]$ \\[2pt]
        
        \textsc{Sortie}: & le nombre minimal de salles qui permet de
        réaliser $H$
      \end{tabular}
      \smallskip
    \end{quote}

    Vous pouvez utiliser une instruction \textbf{trier} qui trie une
    séquence de $n$ éléments, selon l'ordre de votre choix, en temps
    $\O(n \log n)$ dans le pire cas. Analysez le temps d'exécution de
    votre algorithme asymptotiquement dans le pire cas. Vous
    obtiendrez jusqu'à:\medskip
    \begin{itemize}
    \setlength\itemsep{8pt}
      
    \item 7 points si votre algorithme fonctionne en temps
      $\O(n \log n)$,
      
    \item 3{,}5 points si votre algorithme fonctionne en temps $\O(n^2)$.
    \end{itemize}
  \end{enumerate}
\end{question}

\pagebreak

%% Question 3
\SetupExSheets{counter-format=qu[1]:~{notation et analyse asymptotique}}

\begin{question}
  \begin{enumerate}[(a)]
  \setlength\itemsep{15pt}

  \item Considérons \marginpar{\addpoints{3}} un algorithme $\A$ lancé
    sur un ordinateur qui exécute 225~000 millions d'opérations
    élémentaires par seconde. Soit $t(n)$ le temps d'exécution de $\A$
    dans le pire cas. Pour chacune des fonctions suivantes, donnez la
    plus grande valeur (entière) de $n$ où $\A$ termine (à coup sûr)
    en une minute ou moins.
    \begin{multicols}{2}
      \begin{enumerate}[(i)]
      \setlength\itemsep{5pt}
        
      \item $t(n) = n$
        
      \item $t(n) = n \log_2 n$
          
      \item $t(n) = n^2$
          
      \item $t(n) = n^3$
          
      \item $t(n) = 2^n$
        
      \item $t(n) = n!$
      \end{enumerate}
    \end{multicols} 
    Il n'est pas nécessaire de justifier vos réponses.
        
  \item Considérons \marginpar{\addpoints{1}} deux algorithmes $\A$ et
    $\B$. Soit $f(n)$ le temps d'exécution de $\A$ dans le \emph{pire
    cas}, et soit $g(n)$ le temps d'exécution de $\B$ dans le
    \emph{meilleur cas}. Dans les deux scénarios suivants, dites à
    partir de quelle valeur (entière) de $n$ l'algorithme $\A$ est
    \emph{toujours plus rapide} que $\B$:\medskip

    \begin{enumerate}[(i)]
    \setlength\itemsep{5pt}

    \item \makebox[3.45cm][l]{$f(n) = 100000 \cdot n^3$} et $g(n) =
      3^n$,

    \item \makebox[3.45cm][l]{$f(n) = 1000 \cdot n \log_2 n$} et $g(n)
      = n^2$.

    \end{enumerate}
    \bigskip
    Il n'est pas nécessaire de justifier vos réponses.

  \item Montrez \marginpar{\addpoints{3}} que $2(n-4)(n+2)(n+3) + 8
    \log n \in \Theta(n^3)$ sans utiliser les propositions vues en
    classes (donc en identifiant explicitement les constantes
    multiplicatives et seuils).

  \item Ordonnez \marginpar{\addpoints{4}} les fonctions suivantes
    selon la notation $\O$ (de la plus faible vers la plus haute
    complexité):
    \[
    \log_3 (9n^2) + 128!,\qquad
    2^{4 \cdot \log_2(n)} + 100\sqrt{n},\qquad
    \sum_{i=1}^{n} (7i + 2),\qquad
    \sum_{i=1}^n (2^i - 5i),\qquad
    \frac{n}{3} + 1000000
    \]
    Autrement dit, votre réponse devrait être de la forme \og $\O(f_1)
    \subset \O(f_2) \subset \O(f_3) \subset \O(f_4) \subset \O(f_5)$
    \fg. Justifiez \emph{une seule} inclusion en montrant que $f_i \in
    \O(f_{i+1})$ et $f_{i+1} \not\in \O(f_i)$ pour un indice $i$ de
    votre choix.
    
  \item Soit \marginpar{\addpoints{4}} un algorithme $\A$ qui
    fonctionne en temps $f(n) \defeq \sum_{i=0}^n (i / 2^i)$ dans le
    pire cas. Montrez, par induction, que $\displaystyle f(n) =
    (2^{n+1} - n - 2) / 2^n$ pour tout $n \in \N$. Expliquez pourquoi
    $\A$ fonctionne en temps constant.

    \vspace*{75pt}

  \item[\avance{}] Montrez \marginpar{\avancepts{} \addbonus{2.5}}
    qu'il existe des fonctions $f, g \in \F$ telles que $f \not\in
    \O(g)$ et $g \not\in \O(f)$; donc, incomparables selon $\O$.

  \end{enumerate}
\end{question}

\end{document}
