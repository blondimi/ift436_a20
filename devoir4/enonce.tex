\PassOptionsToPackage{table}{xcolor}
\documentclass{article}

%% Mise en pages
\usepackage{fontspec}                 %% >> Pour compiler
\setmainfont{XCharter}                %%                 avec xelatex <<
%\usepackage[utf8]{inputenc}          %% >> Pour compiler avec pdflatex <<
\usepackage[french]{babel}            %% Français
\NoAutoSpacing                        %% Retire l'espacement devant «:»
\usepackage{geometry}                 %% Agrandir les marges
\geometry{left=2.5cm, right=2.5cm, top=2.25cm}
\setlength{\parskip}{\baselineskip}   %% Ajouter espacement entre lignes
\setlength{\parindent}{0pt}           %% Retirer indentation
\usepackage{fancyhdr}                 %% En-têtes/pieds de page
\pagestyle{fancy}                     %%
\fancyhf{}                            %%
\fancyhead[R]{\session}               %%
\fancyhead[L]{\sigle~--~\coursentete} %%
\fancyhead[C]{\evaluation}            %%
\fancyfoot[C]{\thepage}               %%

%% Packages
\usepackage[dvipsnames]{xcolor}       %% Plus de couleurs
\usepackage{amsmath,amssymb}          %% Maths
\usepackage{mathtools}                %% \coloneqq
\usepackage{enumerate}                %% Extension de \begin{enumerate}
\usepackage{exsheets}                 %% Environnements d'exercices
\newcommand*\pointsformat[1]{{\scriptsize#1}}
\SetupExSheets{                       %%
  question/name        = {Question},  %%
  points/name          = {\scriptsize pt/s},
  points/bonus-name    = {\scriptsize pt/s},
}                                     %%
\usepackage{tikz}                     %% Dessins
\usetikzlibrary{shapes, positioning, automata}  %%
\usepackage{hyperref}                 %% Hyperliens
\hypersetup{                          %%
    colorlinks,                       %%
    linkcolor={blue!70!black},        %%
    citecolor={blue!70!black},        %%
    urlcolor={red!70!black}           %%
}                                     %%
\usepackage[noline, onelanguage, french,
  noend]{algorithm2e}                 %% Algorithmes
\usepackage{fontawesome}              %% Icônes
\usepackage{multicol}

%% Virgules francophones dans pointage
%%  (tiré de https://tex.stackexchange.com/a/396982/68991)
\usepackage[locale=FR, detect-weight]{siunitx}
\ExplSyntaxOn
\cs_set_protected:Npn \exsheets_num:n #1
 {
  \num{\fp_eval:n{#1}}
 }
\ExplSyntaxOff

%% Bonus
\newcommand{\avance}{{\large$\bigstar$}}         %% Symbole pour bonus
\newcommand{\avancepts}{{\scriptsize$\bigstar$}} %%

%% Algorithmes
\newcommand{\N}{\mathbb{N}}                      %% Entiers naturels
\newcommand{\R}{\mathbb{R}}                      %% Nombres réels
\newcommand{\mat}[1]{\mathbf{#1}}                %% Matrices
\renewcommand{\O}{\mathcal{O}}                   %% Notation O(·)
\renewcommand{\G}{\mathcal{G}}                   %% Graphe
\newcommand{\T}{\mathcal{T}}                     %% Arbre
\SetKwRepeat{Do}{faire}{tant que}                %% Structure do..while
\newcommand{\algskip}{\vspace{3pt}}              %% Espacement vertical
\definecolor{colComment}{rgb}{0.13, 0.55, 0.13}  %% Commentaires de couleur
\newcommand\commentstyle[1]{\ttfamily\textcolor{colComment}{#1}}
\SetCommentSty{commentstyle}                     %%

%% Couleurs
\colorlet{colEmph}{cyan!80!blue}
\colorlet{colEmph2}{magenta!80!black}
\colorlet{colEmph3}{green!70!black}
\colorlet{colEmph4}{blue!80!purple}
\colorlet{colEmph5}{orange}

\colorlet{colBad}{red}
\colorlet{colWarning}{orange}

%% Commande de titre
\newcommand{\maketitlecustom}{%
  \begin{center}
    \begin{tabular}{c}
      \Large \sigle~--~\cours         \\[3pt]
      \large Université de Sherbrooke \\[10pt]
      \LARGE \textbf{\evaluation}     \\[10pt]
    \end{tabular}
    \begin{tabular}{lp{7.35cm}}
      Enseignant:     & \enseignant   \\
      Date de remise: & \remise       \\
      À réaliser:     & \equipes      \\
      Modalités:      & \modalites    \\
      Bonus:          & \bonust       \\
      Pointage:       & max.\ \pointssum*~points + \bonussum*~points bonus \\
      Consignes & commentez et décrivez les grandes lignes de vos
      algorithmes afin de faciliter leur compréhension
    \end{tabular}
  \end{center}
}

%% Information de l'évaluation
\newcommand{\evaluation}{Devoir 4}
\newcommand{\remise}{mercredi 25 novembre 2020 à 10h29}
\newcommand{\modalites}{remettre en ligne sur \href{http://opus.dinf.usherbrooke.ca/}{Turnin} dans un fichier PDF}
\newcommand{\equipes}{en équipe de deux {\scriptsize ou individuellement}}
\newcommand{\bonust}{les questions bonus sont indiquées par \avance{}}
\newcommand{\sigle}{IFT436}
\newcommand{\cours}{Algorithmes et structures de données}
\newcommand{\coursentete}{Algorithmes et struct.\ de données}
\newcommand{\session}{Automne 2020}
\newcommand{\enseignant}{Michael Blondin}

%% Contenu
\begin{document}

\maketitlecustom

\SetupExSheets{
  question/name        = {Question},
  points/name          = {\scriptsize pt/s},
  points/bonus-name    = {\scriptsize pt/s},
  points/number-format = \pointsformat,
  points/bonus-format  = \pointsformat,
}

\vspace*{-10pt}

%% Commandes pour construire des blocs 3D
%% (Tiré de https://texample.net/tikz/examples/plane-partition/)
%%
% Three counters
\newcounter{x}
\newcounter{y}
\newcounter{z}

% The angles of x,y,z-axes
\newcommand\xaxis{210}
\newcommand\yaxis{-30}
\newcommand\zaxis{90}

% The top side of a cube
\newcommand\topside[3]{
  \fill[fill=colEmph5!25, draw=black,shift={(\xaxis:#1)},shift={(\yaxis:#2)},
  shift={(\zaxis:#3)}] (0,0) -- (30:1) -- (0,1) --(150:1)--(0,0);
}

% The left side of a cube
\newcommand\leftside[3]{
  \fill[fill=colEmph!75, draw=black,shift={(\xaxis:#1)},shift={(\yaxis:#2)},
  shift={(\zaxis:#3)}] (0,0) -- (0,-1) -- (210:1) --(150:1)--(0,0);
}

% The right side of a cube
\newcommand\rightside[3]{
  \fill[fill=colEmph2!75, draw=black,shift={(\xaxis:#1)},shift={(\yaxis:#2)},
  shift={(\zaxis:#3)}] (0,0) -- (30:1) -- (-30:1) --(0,-1)--(0,0);
}

% The cube 
\newcommand\cube[3]{
  \topside{#1}{#2}{#3} \leftside{#1}{#2}{#3} \rightside{#1}{#2}{#3}
}

% Definition of \planepartition
% To draw the following plane partition, just write \planepartition{ {a, b, c}, {d,e} }.
%  a b c
%  d e
\newcommand\planepartition[1]{
 \setcounter{x}{-1}
  \foreach \a in {#1} {
    \addtocounter{x}{1}
    \setcounter{y}{-1}
    \foreach \b in \a {
      \addtocounter{y}{1}
      \setcounter{z}{-1}
      \foreach \c in {1,...,\b} {
        \addtocounter{z}{1}
        \cube{\value{x}}{\value{y}}{\value{z}}
      }
    }
  }
}

%% Directions
\newcommand{\dbas}{\textcolor{colEmph}{\faArrowDown}}
\newcommand{\ddroite}{\textcolor{colEmph2}{\faArrowRight}}

%% Question 1
\begin{question}
  Une \emph{montagne} est une matrice carrée $\mat{M} \in \N^{n \times
    n}$ dont chaque ligne et chaque colonne est ordonnée de façon
  strictement décroissante:\medskip
  \[
  \begin{array}{ccccccc}
    \mat{M}[1, 1] &>& \mat{M}[1, 2] &>& \cdots &>& \mat{M}[1, n] \\[5pt]

    \rotatebox[origin=c]{-90}{$>$} &&
    \rotatebox[origin=c]{-90}{$>$} && &&
    \rotatebox[origin=c]{-90}{$>$} \\[5pt]

    \mat{M}[2, 1] &>& \mat{M}[2, 2] &>& \cdots &>& \mat{M}[2, n] \\[5pt]

    \rotatebox[origin=c]{-90}{$>$} &&
    \rotatebox[origin=c]{-90}{$>$} && &&
    \rotatebox[origin=c]{-90}{$>$} \\[5pt]

    \vdots && \vdots && && \vdots \\[5pt]

    \mat{M}[n, 1] &>& \mat{M}[n, 2] &>& \cdots &>& \mat{M}[n, n]
  \end{array}\smallskip
  \] L'entrée $\mat{M}[i, j]$ indique la hauteur du sol de la montagne
  à la coordonnée $(i, j)$. Le célèbre manchot
  \href{https://fr.wikipedia.org/wiki/Tux}{\emph{Tux~\faLinux}} se
  situe au sommet d'une telle montagne et désire se rendre tout en
  bas. Tux peut descendre d'une position vers une position adjacente
  en glissant sur son ventre le long des parois. Une telle glissade de
  hauteur $h$ dure $f(h)$ secondes, où $f \colon \N_{> 0} \to \R_{>
    0}$ est une fonction qui tient compte de la glace, du plumage, de
  l'accélération, etc. Tux désire connaître la durée d'une descente
  \emph{minimale}. Par exemple, considérons cette montagne:
  \begin{center}
    \vspace*{-15pt}
    \hfill\hfill\hfill\hfill    
      \begin{minipage}[c]{.48\linewidth}
        $\mat{M} = \begin{pmatrix}
          8 & 6 & 4 \\
          7 & 4 & 3 \\
          6 & 2 & 1 \\
        \end{pmatrix}$
        \quad\ $\equiv$
      \end{minipage}\hspace{-3.7cm}
      \begin{minipage}[c]{.48\linewidth}
        \begin{tikzpicture}[thick, scale=0.4, transform shape]
          \planepartition{{8, 6, 4}, {7, 4, 3}, {6, 2, 1}}
          %% Image CC tirée de
          %% https://pixabay.com/fr/vectors/no%C3%ABl-santa-claus-tux-bouchon-161316/
          \node[scale=6] at (0.09, 8.7) {\includegraphics[width=10pt]{tux_xmas}};
          \node[scale=3, red] at (0.15, -1.15) {\faFlag};
        \end{tikzpicture}
      \end{minipage}
  \end{center}

  \vspace*{-15pt}
  Voici la durée de trois des six descentes:
  
  \begin{tabular}{l|l|l}
    \textbf{Chemin} & \textbf{Durée {\small(en secondes)} } &
    \textbf{Ex.} $f(h) = \sqrt{h}$ \\ \hline
    
    \ddroite\ \ddroite\ \dbas\ \dbas &

    $f(8 - 6) + f(6 - 4) + f(4 - 3) + f(3 - 1) = f(2) + f(2) + f(1) +
    f(2)$ & $\approx 5{,}24264$ \\

    \dbas\ \ddroite\ \ddroite\ \dbas &

    $f(8 - 7) + f(7 - 4) + f(4 - 3) + f(3 - 1) = f(1) + f(3) + f(1) +
    f(2)$ & $\approx 5{,}14626$ \\

    \dbas\ \dbas\ \ddroite\ \ddroite &

    $f(8 - 7) + f(7 - 6) + f(6 - 2) + f(2 - 1) = f(1) + f(1) + f(4) +
    f(1)$ & $\phantom{\approx}\; 5\phantom{,00000}$
  \end{tabular}

  En explorant toutes les descentes, on conclurait que la durée
  minimale est de $5$ secondes lorsque $f(n) = \sqrt{n}$.

  \pagebreak

  \begin{enumerate}[(a)]
  \setlength\itemsep{25pt}
    
  \item Si \marginpar{\addpoints{5}} $f(h) = h$, alors on peut
    calculer la durée d'une descente minimale en temps
    $\O(1)$. Pourquoi? Justifiez.

  \item Donnez \marginpar{\addpoints{8}} un algorithme de force brute
    qui identifie la \emph{durée} d'une descente minimale en explorant
    \emph{toutes} les descentes. Votre algorithme fonctionne-t-il en
    temps polynomial par rapport à $n$? Justifiez brièvement.

  \item Donnez \marginpar{\addpoints{8}} un algorithme qui exploite la
    programmation dynamique avec \emph{tableaux} afin d'identifier
    la \emph{durée} d'une descente minimale. Votre algorithme doit
    fonctionner en temps $\O(n^2)$. Quel tableau est calculé par
    l'algorithme sur la montagne ci-dessus avec $f(h) = \sqrt{h}$?
    Vous pouvez donner les valeurs symboliquement ou numériquement
    arrondies à cinq décimaux après la virgule.\label{itm:tux:dyn}

  \item Adaptez \marginpar{\addpoints{4}} votre algorithme obtenu
    en~(\ref{itm:tux:dyn}) afin qu'il retourne une séquence
    de \emph{directions} que Tux peut emprunter afin d'effectuer une
    descente de durée minimale. Par exemple, votre algorithme devrait
    retourner la séquence
    $[\text{\dbas}, \text{\dbas}, \text{\ddroite}, \text{\ddroite}]$
    pour la montagne ci-dessus avec $f(h) = \sqrt{h}$.
  \end{enumerate}

  \vspace*{15pt}
  
  \emph{Clarifications}:\medskip
  \begin{itemize}
    \setlength\itemsep{15pt}

  \item Le sommet et le bas de la montagne se situent respectivement
    aux positions $(1, 1)$ et $(n, n)$;
    
  \item À partir de la position $(i, j)$, Tux peut glisser aux
    positions $(i + 1, j)$ et $(i, j + 1)$, pourvu qu'elles n'excèdent
    pas la dimension de la montagne;

  \item La hauteur de la glissade d'une position $(i, j)$ vers une
    position adjacente $(i', j')$ est $\mat{M}[i, j] - \mat{M}[i',
      j']$;
    
  \item On suppose que l'évaluation de $f(h)$ est une opération
    élémentaire. Imaginez $f$ comme une fonction (pure) offerte par le
    moteur physique d'un jeu vidéo dont vous ne connaissez pas
    l'implémentation.
       
  \end{itemize}

\end{question}

\pagebreak

%% Potions
\newcommand{\potion}[3]{%
  \raisebox{-5pt}{\begin{tikzpicture}[scale=#1, transform shape, #2]%
    \node[inner sep=1pt] at (0, 0) {\faFlask};
    \fill[gray!15, fill opacity=0.6] (0, -0.068) circle (2pt);
    \node[scale=0.45, black] at (0, -0.066) {\textbf{\textsf{#3}}};
  \end{tikzpicture}}%
}
\newcommand{\potionp}[3]{\raisebox{-2pt}{\potion{#1}{#2}{#3}}}

%% Question 2
\begin{question}
  Considérons quatre types de \emph{potions}, représentés par les
  symboles \textsf{a}, \textsf{b}, \textsf{c} et \textsf{x}, que nous
  pouvons combiner. Le résultat des combinaisons des potions est
  décrit par une \og table de multiplication \fg{} comme celle-ci:
  \begin{center}
    \begin{tabular}{c||c|c|c|c}
      \raisebox{-5pt}{\tikz[scale=1.5,
          transform shape]{\node {$\bigotimes$};}} &
      
      \cellcolor{gray!15}\potion{2.0}{magenta}{a} &
      \cellcolor{gray!15}\potion{2.0}{cyan!80!blue}{b} &
      \cellcolor{gray!15}\potion{2.0}{orange}{c} &
      \cellcolor{gray!15}\potion{2.0}{red}{x} \\ \hline \hline
      
      \cellcolor{gray!15}\potion{2.0}{magenta}{a} &
      \potion{2.0}{magenta}{a} &
      \potion{2.0}{cyan!80!blue}{b} &
      \potion{2.0}{red}{x} &
      \potion{2.0}{magenta}{a} \\ \hline
      
      \cellcolor{gray!15}\potion{2.0}{cyan!80!blue}{b} &
      \potion{2.0}{cyan!80!blue}{b} &
      \potion{2.0}{cyan!80!blue}{b} &
      \potion{2.0}{magenta}{a} &
      \potion{2.0}{magenta}{a} \\ \hline

      \cellcolor{gray!15}\potion{2.0}{orange}{c} &
      \potion{2.0}{red}{x} &
      \potion{2.0}{magenta}{a} &
      \potion{2.0}{cyan!80!blue}{b} &
      \potion{2.0}{orange}{c} \\ \hline

      \cellcolor{gray!15}\potion{2.0}{red}{x} &
      \potion{2.0}{magenta}{a} &
      \potion{2.0}{magenta}{a} &
      \potion{2.0}{orange}{c} &
      \potion{2.0}{cyan!80!blue}{b} 
    \end{tabular}
  \end{center}

  On peut combiner une séquence de potions en appliquant l'opération
  $\otimes$. Puisque $\otimes$ n'est pas nécessairement associative,
  l'ordre d'évaluation a une incidence sur le résultat. Par exemple,
  en utilisant l'opération ci-dessus, la séquence $[\textsf{x},
    \textsf{b}, \textsf{c}]$ peut être combinée de deux
  façons:\smallskip
  \begin{alignat*}{2}
  \left(\potionp{2.0}{red}{x} \bigotimes
  \potionp{2.0}{cyan!80!blue}{b}\right) \bigotimes
  \potionp{2.0}{orange}{c}\
  &\raisebox{-3pt}{\scalebox{2}{=}}\
  \potionp{2.0}{magenta}{a} \bigotimes
  \potionp{2.0}{orange}{c}\
  &\raisebox{-3pt}{\scalebox{2}{=}}\
  \potionp{2.0}{red}{x} \\[5pt]
  %
  \potionp{2.0}{red}{x} \bigotimes
  \left(\potionp{2.0}{cyan!80!blue}{b} \bigotimes
  \potionp{2.0}{orange}{c}\right)\
  &\raisebox{-1pt}{\scalebox{2}{=}}\
  \potionp{2.0}{red}{x} \bigotimes
  \potionp{2.0}{magenta}{a}\
  &\raisebox{-1pt}{\scalebox{2}{=}}\
  \potionp{2.0}{magenta}{a}
  \end{alignat*}
  Une druidesse désire combiner consécutivement une séquence de potions
  afin d'obtenir une potion de type \textsf{x}. Autrement dit, elle
  cherche à résoudre ce problème:\medskip
  \begin{quote}
    \vspace*{5pt}
    \begin{tabular}{lp{10.25cm}}
      \textsc{Entrée}: & une séquence $p$ de $n \in \N_{\geq 1}$
      symboles parmi $\{\textsf{a}, \textsf{b}, \textsf{c},
      \textsf{x}\}$, et la table $T$ d'une opération binaire $\otimes$
      sur $\{\textsf{a}, \textsf{b}, \textsf{c}, \textsf{x}\}$ \\[5pt]
      
      \textsc{Sortie}: & est-ce possible de parenthéser l'expression
      $p[1] \otimes p[2] \otimes \cdots \otimes p[n]$ de telle sorte
      qu'elle évalue à \textsf{x}?
    \end{tabular}
    \vspace*{5pt}
  \end{quote}\bigskip

  \begin{enumerate}[(a)]
  \setlength\itemsep{30pt}
    
  \item Dites \marginpar{\addpoints{5}} si les séquences $p =
    [\textsf{x}, \textsf{b}, \textsf{c}, \textsf{a}]$ et $q =
    [\textsf{c}, \textsf{b}, \textsf{c}, \textsf{a}]$ peuvent évaluer
    à \textsf{x} selon la table
    ci-dessus. Justifiez.

  \item Aidez la \marginpar{\addpoints{10}} druidesse en donnant un
    algorithme qui résout son problème. Vous pouvez accéder à la table
    $T$ de l'opération $\otimes$ en temps constant, par exemple: pour
    la table ci-dessus, nous avons $T[\textsf{a}, \textsf{b}] =
    \textsf{b}$ et $T[\textsf{c}, \textsf{a}] = \textsf{x}$. Votre
    algorithme doit fonctionner en temps polynomial par rapport à
    $n$. Expliquez pourquoi c'est le cas.
  \end{enumerate}
\end{question}

\pagebreak

\begin{question}
  \begin{enumerate}[(a)]
  \setlength\itemsep{15pt}

  \item Exécutez \marginpar{\addpoints{5}} l'algorithme de Dijkstra
    sur le graphe pondéré suivant à partir du sommet $a$. Laissez une
    trace de chaque itération de l'algorithme en indiquant les sommets
    marqués et les distances partielles
    identifiées.\bigskip\label{itm:dijkstra}

    \begin{center}
      \begin{tikzpicture}[auto, very thick, node distance=3cm, font=\large,
                          transform shape, scale=0.9]
        \node[state]                               (a) {$a$};
        \node[state, above right=1cm and 2cm of a] (b) {$b$};
        \node[state, below right=1cm and 2cm of a] (c) {$c$};
        \node[state,       right=3cm of b]         (d) {$d$};
        \node[state,       right=3cm of c]         (e) {$e$};
        \node[state, below right=1cm and 2cm of d] (f) {$f$};

        \path[-]
        (a) edge node {$1$} (b)
        (a) edge node[swap] {$4$} (c)
        (b) edge node {$2$} (c)
        (b) edge node {$4$} (d)
        (c) edge node[swap] {$3$} (e)
        (c) edge node[swap] {$1$} (d)
        (d) edge node {$2$} (f)
        (d) edge node {$2$} (e)
        (e) edge node[swap] {$2$} (f)
        ;
      \end{tikzpicture}
    \end{center}

  \item Soit $\G = (V, E)$ un graphe pondéré, non dirigé et
    connexe. Soit $s \in V$ un sommet de $\G$. Un \emph{arbre de plus
      courts chemins (par rapport à $s$)} est un sous-graphe $\T$ de
    $\G$ qui satisfait ces propriétés:\medskip
    
    \begin{itemize}
      \setlength\itemsep{10pt}
      
    \item $\T$ est un arbre couvrant de $\G$,
      
    \item pour tout $v \in V$, l'unique chemin simple de $s$ vers $v$
      dans $\T$ est un plus court chemin de $s$ vers $v$.
    \end{itemize}\bigskip
    
    Par exemple, voici un arbre de plus courts chemins par rapport au
    sommet $s$ de ce graphe:\bigskip\medskip
    \begin{center}
      \begin{tikzpicture}[auto, very thick, node distance=3cm, font=\large,
                          transform shape, scale=0.9]
        \node[state]                 (a) {$s$};
        \node[state, right=2cm of a] (b) {};
        \node[state, below=2cm of a] (c) {};
        \node[state, right=2cm of c] (d) {};
        \node[state, right=2cm of d] (e) {};
        
        \path[-, orange, line width=6pt]
        (a) edge node {} (b)
        (b) edge node {} (d)
        (c) edge node {} (d)
        (d) edge node {} (e)
        ;
        
        \path[-]
        (a) edge node       {$2$} (b)
        (a) edge node[swap] {$7$} (c)
        (b) edge node[swap] {$4$} (c)
        (b) edge node       {$1$} (d)
        (c) edge node[swap] {$2$} (d)
        (d) edge node[swap] {$4$} (e)
        ;
        
      \end{tikzpicture}
    \end{center}\bigskip\medskip
  
    \begin{enumerate}[(i)]
    \setlength\itemsep{20pt}
      
    \item Cet \marginpar{\addpoints{3}} arbre est un arbre couvrant
      \emph{minimal}. On pourrait donc être tenté de croire que les
      deux notions coïncident. Montrez que ce n'est \emph{pas} le
      cas. Autrement dit, identifiez un graphe pondéré $\G$ et un
      sommet $s$ tel qu'un arbre de plus courts chemins par rapport à
      $s$ ne correspond \emph{pas} à un arbre couvrant minimal de
      $\G$. Justifiez.

    \item Identifiez \marginpar{\addpoints{2}} un arbre de plus courts
      chemins par rapport au sommet $a$ du graphe de la
      sous-question~(\ref{itm:dijkstra}).

    \vspace{15pt}
    
    \item[\avance{}] Si \marginpar{\avancepts{} \addbonus{5}} l'on
      connaît des plus courts chemins (non dirigés) à partir de $s$ et
      à partir de $t$, et qu'on retire une arête $e$ du graphe, alors
      il est possible de recalculer la distance entre $s$ et $t$ en
      temps linéaire. Montrez que c'est le cas en donnant un
      algorithme qui résout le problème suivant en temps $\O(|V| +
      |E|)$:

      \begin{quote}
        \vspace*{5pt}
        \begin{tabular}{lp{10cm}}
          \textsc{Entrée}: & un graphe pondéré, dirigé et connexe $\G =
          (V, E)$, une arête $e \in E$, deux sommets $s, t \in V$, et
          deux arbres de plus courts chemins $\T_s$ et $\T_t$ par
          rapport à $s$ et $t$ respectivement \\[5pt]
          
          \textsc{Sortie}: & la distance entre $s$ et $t$ dans le graphe
          $\G$ auquel on retire $e$
        \end{tabular}
        \vspace*{5pt}
      \end{quote}
    \end{enumerate}    
  \end{enumerate}
\end{question}

\end{document}
