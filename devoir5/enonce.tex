\PassOptionsToPackage{table}{xcolor}
\documentclass{article}

%% Mise en pages
\usepackage{fontspec}                 %% >> Pour compiler
\setmainfont{XCharter}                %%                 avec xelatex <<
%\usepackage[utf8]{inputenc}          %% >> Pour compiler avec pdflatex <<
\usepackage[french]{babel}            %% Français
\NoAutoSpacing                        %% Retire l'espacement devant «:»
\usepackage{geometry}                 %% Agrandir les marges
\geometry{left=2.5cm, right=2.5cm, top=2.25cm}
\setlength{\parskip}{\baselineskip}   %% Ajouter espacement entre lignes
\setlength{\parindent}{0pt}           %% Retirer indentation
\usepackage{fancyhdr}                 %% En-têtes/pieds de page
\pagestyle{fancy}                     %%
\fancyhf{}                            %%
\fancyhead[R]{\session}               %%
\fancyhead[L]{\sigle~--~\coursentete} %%
\fancyhead[C]{\evaluation}            %%
\fancyfoot[C]{\thepage}               %%

%% Packages
\usepackage[dvipsnames]{xcolor}       %% Plus de couleurs
\usepackage{amsmath,amssymb}          %% Maths
\usepackage{mathtools}                %% \coloneqq
\usepackage{enumerate}                %% Extension de \begin{enumerate}
\usepackage{exsheets}                 %% Environnements d'exercices
\newcommand*\pointsformat[1]{{\scriptsize#1}}
\SetupExSheets{                       %%
  question/name        = {Question},  %%
  points/name          = {\scriptsize pt/s},
  points/bonus-name    = {\scriptsize pt/s},
}                                     %%
\usepackage{tikz}                     %% Dessins
\usetikzlibrary{shapes, positioning, automata}  %%
\usepackage{hyperref}                 %% Hyperliens
\hypersetup{                          %%
    colorlinks,                       %%
    linkcolor={blue!70!black},        %%
    citecolor={blue!70!black},        %%
    urlcolor={red!70!black}           %%
}                                     %%
\usepackage[noline, onelanguage, french,
  noend]{algorithm2e}                 %% Algorithmes

%% Virgules francophones dans pointage
%%  (tiré de https://tex.stackexchange.com/a/396982/68991)
\usepackage[locale=FR, detect-weight]{siunitx}
\ExplSyntaxOn
\cs_set_protected:Npn \exsheets_num:n #1
 {
  \num{\fp_eval:n{#1}}
 }
\ExplSyntaxOff

%% Bonus
\newcommand{\avance}{{\large$\bigstar$}}         %% Symbole pour bonus
\newcommand{\avancepts}{{\scriptsize$\bigstar$}} %%

%% Algorithmes
\newcommand{\N}{\mathbb{N}}                      %% Entiers naturels
\newcommand{\R}{\mathbb{R}}                      %% Nombres réels
\renewcommand{\O}{\mathcal{O}}                   %% Notation O(·)
\SetKwRepeat{Do}{faire}{tant que}                %% Structure do..while
\SetKwFor{RepTimes}{faire}{fois}{}               %% Structure faire n fois
\definecolor{colComment}{rgb}{0.13, 0.55, 0.13}  %% Commentaires de couleur
\newcommand\commentstyle[1]{\ttfamily\textcolor{colComment}{#1}}
\SetCommentSty{commentstyle}                     %%
\newcommand{\algskip}{\vspace{3pt}}              %% Espacement vertical
\newcommand{\algdelim}{\texttt{:}}               %%
\newcommand{\emptyseq}{[\,]}                     %% Séquence vide

%% Commande de titre
\newcommand{\maketitlecustom}{%
  \begin{center}
    \begin{tabular}{c}
      \Large \sigle~--~\cours         \\[3pt]
      \large Université de Sherbrooke \\[10pt]
      \LARGE \textbf{\evaluation}     \\[10pt]
    \end{tabular}
    \begin{tabular}{lp{7.35cm}}
      Enseignant:     & \enseignant   \\
      Date de remise: & \remise       \\
      À réaliser:     & \equipes      \\
      Modalités:      & \modalites    \\
      Bonus:          & \bonust       \\
      Pointage:       & max.\ \pointssum*~points + \bonussum*~points bonus \\
      Consignes & commentez et décrivez les grandes lignes de vos
      algorithmes afin de faciliter leur compréhension
    \end{tabular}
  \end{center}
}

%% Information de l'évaluation
\newcommand{\evaluation}{Devoir 5}
\newcommand{\remise}{mercredi 2 décembre 2020 à 10h29}
\newcommand{\modalites}{remettre en ligne sur \href{http://opus.dinf.usherbrooke.ca/}{Turnin} dans un fichier PDF}
\newcommand{\equipes}{en équipe de deux {\scriptsize ou individuellement}}
\newcommand{\bonust}{les questions bonus sont indiquées par \avance{}}
\newcommand{\sigle}{IFT436}
\newcommand{\cours}{Algorithmes et structures de données}
\newcommand{\coursentete}{Algorithmes et struct.\ de données}
\newcommand{\session}{Automne 2020}
\newcommand{\enseignant}{Michael Blondin}

%% Contenu
\begin{document}

\maketitlecustom

\SetupExSheets{
  question/name        = {Question},
  points/name          = {\scriptsize pt/s},
  points/bonus-name    = {\scriptsize pt/s},
  points/number-format = \pointsformat,
  points/bonus-format  = \pointsformat,
}

\vspace*{1cm}

\newcommand{\rang}[1]{\mathrm{rang}(#1)}

Le pivot idéal pour le tri rapide est la médiane. En théorie, on peut
calculer la médiane d'une séquence en temps linéaire, mais cela
s'avère peu efficace en pratique. Une autre approche consiste à
choisir un élément à une distance raisonnable de la médiane, comme un
élément compris entre le premier et troisième quartile.

Soit $s$ une séquence de $n$ éléments comparables distincts. Le
\emph{rang} d'un élément $x \in s$, dénoté $\rang{x}$, correspond à la
position de $x$ dans $s$ ordonnée en ordre croissant (en comptant à
partir de $1$). Nous disons qu'un élément $x \in s$
est \emph{raisonnable} si $$\lceil n / 4\rceil < \rang{x} \leq \lfloor
3n / 4 \rfloor.$$ Par exemple, considérons la séquence $s = [50, 13,
20, 67, 41, 89, 70, 30]$. Nous avons $\rang{13} = 1$, $\rang{67} = 6$
et $\rang{89} = 8$. De plus, les pivots raisonnables de $s$ sont $30$,
$41$, $50$ et $67$.

Afin d'identifier un pivot raisonnable, une approche probabiliste
consiste à choisir $k$ éléments aléatoirement (de façon uniforme),
puis de retourner la médiane de ces $k$ éléments, où $k \in \N$ est un
paramètre impair:

{
  \setlength{\interspacetitleruled}{0pt}%
  \setlength{\algotitleheightrule}{0pt}%
  \begin{algorithm}[H]
    \DontPrintSemicolon
    \SetKwFunction{pseudomed}{$\texttt{pseudomed}_k\!$}
    \SetKwProg{myproc}{}{}{}
    \KwIn{séquence $s$ de $n \in \N_{\geq 3}$ éléments comparables distincts}

    \KwResult{un élément raisonnable $x \in s$}
    \algskip
    
    \myproc{\pseudomed{$s$}\algdelim}{
      $t \leftarrow \emptyseq$\;
      \algskip

      \RepTimes{$k$}{
        \textbf{choisir} $i \in [1..n]$ aléatoirement de façon uniforme\;
        \textbf{ajouter} $s[i]$ \textbf{à} $t$\;
      }

      \algskip
      \textbf{trier} $t$\;
      \algskip
      
      \Return{$t[\lceil k / 2 \rceil]$}\;
    }
  \end{algorithm}
}

\vspace*{8pt}

{\small
\emph{Remarques}:\medskip
\begin{itemize}
  \setlength\itemsep{8pt}
  
\item la constante $k$ ne fait \emph{pas} partie de l'entrée, il
  s'agit d'un paramètre fixe. Autrement dit,
  $\texttt{pseudomed}_1, \allowbreak \texttt{pseudomed}_3, \allowbreak \texttt{pseudomed}_5, \texttt{pseudomed}_7, \ldots$
  sont tous des algorithmes différents;

\item le choix d'un nombre aléatoire est ici une
  opération élémentaire qui s'effectue \emph{toujours} en temps
  constant.
\end{itemize}
}

\pagebreak

\begin{question}
  \begin{enumerate}[(a)]
    \setlength\itemsep{25pt}

  \item L'algorithme \marginpar{\addpoints{3}} \pseudomed est-il de
    Las Vegas ou de Monte Carlo? Justifiez.

  \item Quelle \marginpar{\addpoints{5}} est la probabilité que
    $\texttt{pseudomed}_3$ retourne la médiane de $s = [42, 9000, 0]$?
    Justifiez.\medskip

     {\small\hfill\emph{Indice}: on peut procéder par comptage plutôt
     que par énumération exhaustive.} \\[-15pt]

  \item[] \hspace*{-17pt}\underline{\emph{Pour les sous-questions
      suivantes, supposez que $n$ est divisible par $4$ (cela
      simplifie les calculs).}} \\[-20pt]    

  \item Donnez \marginpar{\addpoints{8}} la probabilité d'erreur de
     $\texttt{pseudomed}_1$ et de $\texttt{pseudomed}_3$, puis
     généralisez en donnant une expression symbolique qui décrit la
     probabilité d'erreur de
     $\texttt{pseudomed}_k$. Justifiez.\medskip

     {\small\hfill\emph{Indice}: considérez trois types d'éléments:
     raisonnables, déraisonnables de gauche et déraisonnables de droite; }
     \\[-15pt]
     
     {\small\hfill pensez au coefficient binomial.}

  \item Donnez \marginpar{\addpoints{4}} un algorithme (déterministe)
    qui détermine en temps $\O(|s|)$ si un élément $x \in s$ est
    raisonnable.\label{qu:raisonnable}

  \item La procédure \marginpar{\addpoints{5}} ci-dessous exécute
    $\texttt{pseudomed}_3$ jusqu'à ce qu'un élément raisonnable soit
    identifié. Ainsi, sa valeur de retour est toujours correcte. Quel
    est son \emph{temps espéré} sous notation $\O$? Justifiez.

    \begin{minipage}[t]{0.9575\textwidth}
    {
      \setlength{\interspacetitleruled}{0pt}%
      \setlength{\algotitleheightrule}{0pt}%
      \begin{algorithm}[H]
        \DontPrintSemicolon
        \SetKwProg{myproc}{}{}{}
        \KwIn{séquence $s$ de $n \in \N_{\geq 3}$
              éléments comparables distincts}
        \KwResult{un élément raisonnable $x \in s$}
        \algskip

        \Do{$x$ n'est pas raisonnable
          \tcp*[f]{déterminé avec l'algo.\ de la sous-question~(d)}}{
          $x \leftarrow \texttt{pseudomed}_3(s)$\;
        }
        
        \algskip
        \Return{$x$}
        
      \end{algorithm}
    }
    \end{minipage}

    \vspace*{80pt}
    
  \item[\avance{}] Montrez \marginpar{\avancepts{} \addbonus{2.5}} qu'il
    existe une valeur de $k$ pour laquelle la probabilité d'erreur de
    \pseudomed est $\leq 1 / 2^{275}$.
    
  \end{enumerate}
\end{question}

\end{document}
