\documentclass{article}

%% Packages
\usepackage[utf8]{inputenc}
\usepackage[french]{babel}
\usepackage{amsmath,amssymb}
\usepackage{enumerate}
\usepackage[noline, onelanguage, french, noend]{algorithm2e}

%% Macros
\newcommand{\N}{\mathbb{N}}
\renewcommand{\O}{\mathcal{O}}
\SetKwFor{RepTimes}{faire}{fois}{}
\newcommand{\algdelim}{\texttt{:}}
\newcommand{\algskip}{\vspace*{3pt}}
\newcommand{\emptyseq}{[\,]}
\newcommand{\lto}{\,{:}\,}

%% Titre
\title{IFT436: devoir 5}
\author{Foo McBar}
\date{2 décembre 2020}

\begin{document}
\maketitle

%% Contenu
\section*{Question 1}

\begin{enumerate}[(a)]
\item Oui/non car...
  
\item \[\frac{1}{2^n}\]

\item \[{10 \choose 5} \cdot 100!\]
    
\item ...

  \begin{algorithm}[H]
    \DontPrintSemicolon
    \SetKwFunction{pseudomed}{$\texttt{pseudomed}_k\!$}
    \SetKwProg{myproc}{}{}{}
    \KwIn{séquence $s$ de $n \in \N_{\geq 3}$ éléments comparables distincts}

    \KwResult{un élément raisonnable $x \in s$}
    \algskip
    
    \myproc{\pseudomed{$s$}\algdelim}{
      $t \leftarrow \emptyseq$\;
      \algskip

      \RepTimes{$k$}{
        \textbf{choisir} $i \in [1..n]$ aléatoirement de façon uniforme\;
        \textbf{ajouter} $s[i]$ \textbf{à} $t$\;
      }

      \algskip
      \textbf{trier} $t$\;
      \algskip
      
      \Return{$t[\lceil k / 2 \rceil]$}\;
    }
  \end{algorithm}  

\item $\O(n^3)$
\end{enumerate}

\end{document}
